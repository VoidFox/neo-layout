\documentclass{scrartcl}

\usepackage[ngerman]{babel}
\usepackage{hyperref}

\usepackage{fontspec}
\usepackage{xunicode}
\usepackage{xltxtra}

\setmainfont{Linux Libertine O} % Constantia, Gentium, Linux Libertine O, YanoneKaffeesatz, ComicJens …

%\setkomafont{disposition}{\fontspec{Gentium}} % Benutzt eine Schrift für alle Gliederungsebenen

% Aber wir wollen sie lieber manuell konfigurieren:
\setkomafont{section}{\fontspec{Linux Libertine O}\Huge\textbf}
\setkomafont{subsection}{\fontspec[Color=FF2010B9]{Linux Libertine O}\Large}



\begin{document}


\section*{Liesmich}
Zuerſt ſollte man ſich die Schrift »Linux Libertine O« im OpenType-Format herunterladen. Dieſe muſs dann nur noch auf \emph{ganz normale, betriebsſyſtemspezifiſche} Art und Weiſe im jeweiligen Betriebsſyſtem (Windows, Linux, Apple) inſtalliert werden. Anſchließend kann dieſe Beiſpieldatei problemlos mit »xelatex LiesMich-LinuxLibertine.tex« kompiliert werden (eine moderne Mik\TeX- oder \TeX Life-Distribution vorausgeſetzt).


\subsection*{Ein paar nette \XeLaTeX-Spielereien}
¡Willkommen zu \XeLaTeX, dem \TeX-Derivat der Zukunft! Die Schriften des Betriebsſyſtems werden \emph{nahtlos} und \emph{vollautomatiſch} mitſamt aller Schriftſchnitte in \XeTeX\ integriert. Oft werden ſo ſogar erweiterte OpenType-Features anſprechbar, die in Word, OpenOffice etc.\ noch gar nicht unterſtützt werden.

Dies betrifft beispielsweise dichtengleiche Zahlen wie {\addfontfeatures{Numbers={Lining,Monospaced,SlashedZero}}1234,50} (vs.\ 1234,50), Mediäval- oder Minuskelziffern wie {\addfontfeature{Numbers=OldStyle}0123456789}, normale wie ſeltene {\addfontfeatures{Ligatures={Rare,Historical}}fluffige starke Ligaturenſchätze wie »ct«}, aber auch {\addfontfeatures{Letters=SmallCaps,LetterSpace=6.0}ECHTE KAPITÄLCHEN}.

Wird ſo ein Feature im aktuellen Font hingegen nicht unterſtützt, gibt \XeTeX\ eine Warnmeldung in die Log-Datei aus.

Zudem arbeitet \XeLaTeX\ ſtandardmäßig mit der UTF-8-Kodierung, was die direkte Eingabe beliebiger Unicode-Zeichen möglich macht: α,β,γ, °№§, ¹²³, ♀⚥♂, MAẞEN. Ältere \LaTeX-Syntax kann hingegen manchmal zu unerwarteten (aber logiſchen) Problemen führen, vergleiche etwa – (Gedankenstrich) vs.\ -- (Divis-Divis) vs.\ {\addfontfeatures{Mapping=tex-text}-- (Divis-Divis, \LaTeX-Kompatibilitätsmodus)}.


\subsection*{Wo Licht iſt, iſt auch Schatten}
Die Unterſtützung des Mathematik-Modus iſt vorhanden, aber noch experimentell und deshalb in einem ſeparaten Packet ausgelagert. Zudem werden die typographiſchen Feinheiten des Microtype Packetes (optischer Randausgleich etc.) derzeitig noch nicht unterſtützt.


\subsection{Links}
\begin{description}
\item[\url{http://scripts.sil.org/xetex/}:] Die offizielle englischsrachige \XeTeX−Homepage.
\item[\url{http://linuxlibertine.sourceforge.net/XeTex/Libertine-XeTex-DE.pdf}:] Verwendung der Schrift »Linux Libertine« mit \XeTeX\ – Beihaltet Konfigurationsbeiſpiele und zeigt Vorteile von \XeTeX\ gegenüber klaſſiſchem \LaTeX\ auf.
\item[\url{http://xml.web.cern.ch/XML/lgc2/xetexmain.pdf}:] Eine sehr ausführliche engliſchſprachige XeTeX-Einführung, die neben dem praktiſchen Teil auch die \XeTeX\ zu Grunde liegenden Softwareſtandards (Unicode, OpenType, …) behandelt.
\end{description}


\input{Wikipedia-Artikel.utf8}



\end{document}
