%!TEX TS-program = xelatex
%!TEX encoding = UTF-8 Unicode
%% Die Datei ist also UTF-8-kodiert, und mit "xelatex geek.tex" zu kompilieren

%% Koma-Script
\documentclass[fontsize=10pt]{scrartcl} % oder scrbook?

%% XeTeX
\usepackage{xltxtra}

% Grundschrift
\setmainfont{Gentium}

%% Sprachen
\usepackage{polyglossia}
\setmainlanguage[spelling=new, latesthyphen=true]{german}
\setotherlanguage[variant=ancient]{greek}

% Schrift für Griechisch
\newfontfamily\greekfont[ExternalLocation]{SBL_grk.ttf}

% Links erzeugen
\usepackage{hyperref}


\begin{document}


\title{Deutsch/Griechische Texte in \XeTeX}
\author{\textsc{Xena} von Amphipolis}
\date{Stand: \today\ (oder auch \textgreek{\today})}
\maketitle

\tableofcontents

\section{Einleitung}
Die griechischen Zeichen lassen sich mit \XeTeX\ so erzeugen: \textgreek{ἡ Ἑλληνικὴ γλῶσσα} sowie \textgreek{ὦ, ᾖς, ᾖ, ὦμεν, ἦτε, ὦσι(ν)} und \textgreek{α ἁᾳ βγδεζη ϑ θι ϰ κλμνξοπ ϱ ρσςτυφ ϕ χψω ΓΔΘΛΞΠΣΦΨΩ} …

Auch die \emph{Gentium}\footnote{\url{http://www.sil.org/~gaultney/Gentium/}} bringt einige griechische Glyphen wie α, δ oder Ψ mit, aber wohl nicht ganz so viele wie die \emph{SBL Greek}\footnote{\url{http://www.sbl-site.org/educational/BiblicalFonts_SBLGreek.aspx}}. Und es ist sowieso schöner, für eine fremde Sprache konsequent eine fremde Schrift zu nutzen.

Graf Zahl zählt jetzt auch auf griechisch: \textgreek{1863=\greeknumeral{1863}} Neolinge, \textgreek{1864=\greeknumeral{1864}} Neolinge, … und um auch einmal einen längeren Textausſchnitt aus Homers Ilias zu zitieren:

\newcommand{\ilias}{μῆνιν ἄειδε θεὰ Πηληϊάδεω Ἀχιλῆος οὐλομένην, ἣ μυρί' Ἀχαιοῖς ἄλγε'
ἔθηκε, πολλὰς δ' ἰφθίμους ψυχὰς Ἄϊδι προί̈αψεν ἡρώων, αὐτοὺς δὲ ἑλώρια
τεῦχε κύνεσσιν οἰωνοῖσί τε πᾶσι, Διὸς δ' ἐτελείετο βουλή, ἐξ οὗ δὴ τὰ
πρῶτα διαστήτην ἐρίσαντε Ἀτρεί̈δης τε ἄναξ ἀνδρῶν καὶ δῖος Ἀχιλλεύς.}

\begin{quote}
\begin{greek}
\ilias
\end{greek}
\end{quote}

Und was wäre dieses Dokument, ohne auch noch etwas mit \emph{fontspec} herumzuspielen:
\begin{quote}
\begin{greek}
\addfontfeatures{Variant=4}
\ilias
\end{greek}
\end{quote}

\section{\texorpdfstring{\textgreek{ἀληθινόν} im Text und im Inhaltsverzeichnis}{Umschrift für den PDF-Link}}
¡Bei mir kompiliert’s ohne Fehler!


\end{document}