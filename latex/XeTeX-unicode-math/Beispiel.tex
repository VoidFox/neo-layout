\documentclass{article}

\usepackage{xltxtra} % lädt auch fixltx2e, etex, fontspec, xunicode …

\setmainfont{Cambria}
\setmathfont{Cambria Math}

\usepackage{polyglossia}
\setdefaultlanguage[spelling=new, latesthyphen=true]{german}

% \mathrel \mathord \mathalpha …
\UnicodeMathSymbol{"02026}{\ldots    }{\mathpunct}{Horizontal ellipsis}
\UnicodeMathSymbol{"000E8}{\grave{e} }{}{LATIN SMALL LETTER E WITH GRAVE}
\UnicodeMathSymbol{"000E9}{\acute{e} }{}{LATIN SMALL LETTER E WITH ACUTE}
\UnicodeMathSymbol{"00117}{\dot{e}   }{}{LATIN SMALL LETTER E WITH DOT ABOVE}
\UnicodeMathSymbol{"00113}{\bar{e}   }{}{LATIN SMALL LETTER E WITH MACRON}
%\UnicodeMathSymbol{"01E9E}{\SS      }{}{LATIN CAPITAL LETTER SHARP S}
%\UnicodeMathSymbol{"003A3}{\sum     }{\mathop}{GREEK CAPITAL LETTER SIGMA}



\begin{document}

\title{Das Packet \emph{unicode-math«}}


\section{Unicode im Mathematik-Modus benutzen}
\XeLaTeX\ versteht die direkte Eingabe von Unicode-Zeichen wie äöüß, αβδε oder ∞∫ – ¡und das mit Hilfe des Paketes \emph{unicode-math} auch im Mathematik-Modus! Man kann also anstatt wie bisher
\[ \forall \varepsilon>0 \exists \delta>0 \forall 1 \geq\Delta\geq0: \infty>\varepsilon,\delta,\delta+\Delta>0, \mathbf{0\leq\Delta\leq1} \]
auch einfach
\[ ∀ ε>0 ∃ δ>0 ∀1≥Δ≥0: ∞>ε,δ,δ+Δ>0, \mathbf{0≤Δ≤1} \]
eingeben.

\section{Diverse Tests}
Interpretation von doppeldeutigen Zeichen:
\[ Σ=\Sigma ≠ \sum, Π=\Pi ≠ \prod, \sum_{δ=0}^Ψ, ∫_a^b, \int_aˆb… \]

Behandlung in der aktuellen Schrift unbekannter Zeichen: PREUẞEN vs.
\[ PREUẞEN \]

Auszeichnungen:
\[ \sin(π) ≠ E, \mathbf{\sin(π) ≠ E,}  \]

Diakritische Zeichen:
\[ α_1, α_2, …, α_n, é è ē ė \hat{a}, ℝ, ℂ, ⇋↓⇌←†→↑≪∩≫⊂∊⊃≤∪≥⊃∊⊂≠±× \]


\end{document}