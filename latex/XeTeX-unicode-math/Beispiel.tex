\documentclass{article}

\usepackage{xltxtra} % lädt auch fixltx2e, etex, fontspec, xunicode …

% Grundschrift für den normalen Text:
\setmainfont{Cambria}

% Lade unicode-math und einen passende Mathematikschrift:
\usepackage{unicode-math}
\setmathfont{Cambria Math}

% Statt babel:
\usepackage{polyglossia}
\setdefaultlanguage[spelling=new, latesthyphen=true]{german}


\begin{document}

\title{Einstieg in das \emph{unicode-math} Paket}
\author{http://neo-layout.org/}
\maketitle

\section{Unicode im Mathematik-Modus benutzen}
\XeLaTeX\ versteht die direkte Eingabe von Unicode-Zeichen wie äöüß, αβδε oder ∞∫ – ¡und das mit Hilfe des Paketes \emph{unicode-math} auch im Mathematik-Modus! Man kann also anstatt wie bisher
\[ \forall \epsilon>0 \exists \delta>0 \forall 1 \geq\Delta\geq0: \infty>\epsilon,\delta,\delta+\Delta>0, \mathbf{0\leq\Delta\leq1} \]
auch einfach
\[ ∀ ε>0 ∃ δ>0 ∀1≥Δ≥0: ∞>ε,δ,δ+Δ>0, \mathbf{0≤Δ≤1} \]
eingeben.

\section{Diverse Tests}
Interpretation von doppeldeutigen Zeichen:
\[ Σ=\Sigma ≠ \sum, Π=\Pi ≠ \prod, \sum_{δ=0}^Ψ, ∫_a^b, \int_aˆb… \]

Behandlung in der aktuellen Schrift unbekannter Zeichen: PREUẞEN vs.
\[ PREUẞEN \]

Auszeichnungen:
\[ \sin(π) ≠ E, \mathbf{\sin(π) ≠ E,}  \]

Diakritische Zeichen:
\[ α_1, α_2, …, α_n, é è ē ė \hat{a}, ℝ, ℂ, ⇋↓⇌←†→↑≪∩≫⊂∊⊃≤∪≥⊃∊⊂≠±× \]


\end{document}